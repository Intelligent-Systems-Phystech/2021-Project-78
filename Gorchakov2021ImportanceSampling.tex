\documentclass[12pt, twoside]{article}
\usepackage{jmlda}
\newcommand{\hdir}{.}

\begin{document}

\title
    [] % краткое название; не нужно, если полное название влезает в~колонтитул
    {}
\author
    [В.\,В. ~Горчаков] % список авторов (не более трех) для колонтитула; не нужен, если основной список влезает в колонтитул
    {В.\,В.~Горчаков, Ю.\,.~Максимов} % основной список авторов, выводимый в оглавление
    [В.\,В.~Горчаков, Ю.\,.~Максимов] % список авторов, выводимый в заголовок; не нужен, если он не отличается от основного
\email
   {gorchakov.v@phystech.edu, yurymaximov@gmail.com}
%\thanks
%    {Работа выполнена при
%     %частичной
%     финансовой поддержке РФФИ, проекты \No\ \No 00-00-00000 и 00-00-00001.}
%\organization
%    {$^1$Организация, адрес; $^2$Организация, адрес}
\abstract
  {Работа посвящена использованию метода сэмплирования по важности в оптимизационных задачах с вероятностными ограничениями.Исследуются методы вычисления оптимальных сэмплеров для снижения числа сценариев. Решается задача нахождения минимума выпуклой функции при вероятностных ограничениях  с получением численной оценки эффективности применения семплирования в этой задаче.
  
  

\bigskip
\noindent
\textbf{Ключевые слова}: \emph {}
}

%данные поля заполняются редакцией журнала
\doi{}
\receivedRus{}
\receivedEng{}

\maketitle
\linenumbers

\section{Введение}


Задачи оптимизации с вероятностными ограничениями часто встречаются в инженерной практике. Например, задача минимизации генерации энергии в энергетических сетях, со (случайным образом флуктуирующими) возобновляемыми источниками энергии. При этом необходимо выполнение ограничений безопасности: напряжения у генераторов и потребителей, а также токи на линиях должны быть меньше определенных порогов. Вместе с тем, даже в самых простых ситуациях задача не может быть разрешена точно. Самый известный подход, это методы оптимизации с вероятностными ограничениями, которые часто дают хорошее приближение. Еще одним подходом, который позволяет добиться неплохого качества, является семплирование режимов работы сети и решения задачи на наборе данных задачи классификации: отделение плохих режимов от хороших с заданной ошибкой второго рода. Вместе с тем, для достаточно точного решения, требуется очень большой объем данных, что часто делает задачу численно не эффективной. Мы предлагаем использовать “семплирование по важности” для уменьшения числа сценариев. 

Идея семплирования по важности состоит в том, чтобы заменять выборку из номинального решения, которое зачастую в силу редкости плохих событий не является информативным, на синтетическое распределение, которое семплирует выборку в окрестности плохих событий. 

Существуют работы, в которых применяется сэмплирование по важности для набора редких объектов. В этой работе предлагается способ для вычисления оптимальных сэмплеров, которые позволят существенно снизить число сценариев.

Решение поставленной задачи позволяет разработать алгоритмы, применяемые в работе с возобновляемыми источниками энергии и энергетических сетях. Результаты важны с точки зрения инженерной практики для выполнения ограничений безопасности на линиях электропередач.

\section{Название раздела}
Данный документ демонстрирует оформление статьи,
подаваемой в электронную систему подачи статей \url{http://jmlda.org/papers} для публикации в журнале <<Машинное обучение и анализ данных>>.
Более подробные инструкции по~стилевому файлу \texttt{jmlda.sty} и~использованию издательской системы \LaTeXe\
находятся в~документе \texttt{authors-guide.pdf}.
Работу над статьёй удобно начинать с~правки \TeX-файла данного документа.

Обращаем внимание, что данный документ должен быть сохранен в кодировке~\verb'UTF-8 without BOM'.
Для смены кодировки рекомендуется пользоваться текстовыми редакторами \verb'Sublime Text' или \verb'Notepad++'.

\paragraph{Название параграфа}
Разделы и~параграфы, за исключением списков литературы, нумеруются.

\section{Заключение}
Желательно, чтобы этот раздел был, причём он не~должен дословно повторять аннотацию.
Обычно здесь отмечают, каких результатов удалось добиться, какие проблемы остались открытыми.

%%%% если имеется doi цитируемого источника, необходимо его указать, см. пример в \bibitem{article}
%%%% DOI публикации, зарегистрированной в системе Crossref, можно получить по адресу http://www.crossref.org/guestquery/

%%%% если имеется doi цитируемого источника, необходимо его указать, см. пример в \bibitem{article}
%%%% DOI публикации, зарегистрированной в системе Crossref, можно получить по адресу http://www.crossref.org/guestquery/.

%%%% если имеется doi цитируемого источника, необходимо его указать, см. пример в \bibitem{article}
%%%% DOI публикации, зарегистрированной в системе Crossref, можно получить по адресу http://www.crossref.org/guestquery/
\begin{thebibliography}{99}


 \bibitem{article}
    \BibAuthor{Art B.~Owen, Y.~Maximov, M.~Chertkov}
   Importance Sampling for the Union of Rare Events with Applications to Power Systems~//
    \BibJournal{},
	
 \bibitem{article} 
    \BibAuthor{A. Nemirovski}
   On safe tractable approximations of chance constraints~//
    \BibJournal{}
		

 	
\end{thebibliography}

%%%% если имеется doi цитируемого источника, необходимо его указать, см. пример в \bibitem{article}
%%%% DOI публикации, зарегистрированной в системе Crossref, можно получить по адресу http://www.crossref.org/guestquery/.

\end{document} 